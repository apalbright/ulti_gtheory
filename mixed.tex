\documentclass{article}
\usepackage{multirow,array}
\usepackage{amsmath}
\usepackage{amssymb}
%\usepackage{epsf,epsfig,epic,eepic}
\usepackage{graphicx}
\begin{document}

\section{T2 pure \& T1 mix?}
This is not possible since the indifference condition fails to hold [both if T2 plays W and if T2 plays S]:

$$u_1(W,W)=4>2=u_1(S,W)$$ $$u_1(S,S)=2>-2=u_1(W,S)$$

\section{T1 pure \& T2 mix?}
This is not possible since the indifference condition fails to hold [both if T1 plays W and if T1 plays S]:

$$u_2(W,S)=2>-2=u_2(W,W)$$ $$u_2(S,W)=-2>-4=u_2(S,S)$$

\section{Both mix?}
We know that since we have a finite game, there must exist a Nash equilibrium. Since we have shown that neither team can play pure strategies, it must be the case that both mix. 

T1 must be indifferent between playing W and S. Say T1 plays W with probability $\alpha$ and S with probability $1-\alpha$, then we set
$$4\alpha - 2 (1-\alpha)=2\alpha + 2(1-\alpha)$$
$$\alpha=\frac{2}{3}$$

Moreover, T2 must also be indifferent between playing W and S. Say T2 plays W with probability $\beta$ and S with probability $1-\beta$, then we set
$$-2\beta - 2 (1-\beta)=2\beta - 4(1-\beta)$$
$$\beta=\frac{1}{3}$$

\underline{Thus, the mixed Nash equilibrium is: ($\frac{2}{3}W \oplus \frac{1}{3}S$, $\frac{1}{3}W \oplus \frac{2}{3}S$)}

\end{document}