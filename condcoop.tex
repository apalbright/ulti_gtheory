\documentclass{article}
\usepackage{multirow,array}
\usepackage{amsmath}
\usepackage{amssymb}
%\usepackage{epsf,epsfig,epic,eepic}
\usepackage{graphicx}
\begin{document}

\section*{Grim Trigger Strategy}
Teams play weak lines until one defects and plays a strong line. After that, both teams play strong lines from then on. Let's show that in the repeated game setting this strategy can give us efficiency in subgame perfect equilibrium.

\section{Other team following strategy and no strong lines yet...}
In this set up, the other team will always play a weak line as long as we (as the other team) do so as well.

So, if we follow strategy our payoff is the following with discounting of $\delta$ as discount factor:

$$1+\delta+\delta^2+...=\frac{1}{1-\delta}$$

If we choose to play a strong line instead, then the flow of payoffs will be:
$$2-\delta-\delta^2+...=\frac{-1}{1-\delta}+3$$

We will choose to continue to play weak lines if $\frac{1}{1-\delta}>\frac{-1}{1-\delta}+3 \Rightarrow \delta>\frac{1}{3}$


\section{Other team following strategy and there has been a strong line...}
In this set up, the other team will always play a strong line since a strong line has been played.

So, if we follow the strategy (play strong in response) our payoff is:

$$-1-\delta-\delta^2+...=\frac{-1}{1-\delta}$$

If we choose to play a weak line at any time period, the sum total of the flow payoffs will be lower since we will receive -2 in that period instead of -1. So, we don't gain from deviating from the specified strategy. (We will play strong from now on.)

\bigskip
\underline{Thus, if $\delta>\frac{1}{3}$, this strategy is a subgame perfect equilibrium!}

\end{document}